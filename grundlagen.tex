\section{Grundlagen}
	\subsection{Absoluter und relativer Fehler}
		Um Fehler mathematisch exakt quantifizieren zu können, müssen wir messen. Für das
		Messen benötigen wir eine Norm:
		
		\subsubsection{Norm}
			Eine Norm auf einem $ \mathbb{K}-$Vektorraum V (V ist entweder $ \mathbb{R} $ oder $ \mathbb{C} $) ist eine Abbildung $ \Norm{\cdot} $ mit folgenden Eigenschaften:
			\begin{itemize}
				\item Definitheit: $\forall v \in V: \Norm{v} \geq 0 $ und dazu $ \Norm{v} = 0 \leftrightarrow v = 0 $
				\item Homogenität: $\forall a \in \mathbb{K}, v \in V: ||av|| = |a| \cdot ||v|| $
				\item Dreiecksungleichung: $ \forall v,w \in V: ||v+w|| \leq ||v|| + ||w||$
			\end{itemize}